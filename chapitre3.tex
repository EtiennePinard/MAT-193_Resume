\section{Chapitre 3: Espace vectorielle de dimension finie}
\begin{definition}
      Un ensemble $V$ est un espace vectorielle sur un corps $F$
      si \begin{enumerate}[1)]
            \item $V$ est fermé sous l'addition c-à-d, $
                        \forall \ (v_1, v_2 \in V), \ v_1 + v_2 \in V
                  $
            \item $V$ est fermé sous la multication par un scalaire c-à-d, $
                        \forall \ (v \in V, \ \alpha \in F), \ \alpha v \in V
                  $
      \end{enumerate}
      Les éléments de $V$ sont appellés \guillemetleft \ vecteurs \guillemetright.
\end{definition}
\begin{remark}
      L'espace vectorielle le plus simple est $V = \{ 0_v \}$, ou $0_v$ est l'élément nulle.
\end{remark}
\begin{definition}
      Un scalaire $\alpha$ est un élément du corps $F$ associé à l'espace vectorielle.
\end{definition}
\noindent
Dans notre cas, ce corps est soit les nombres réels $\mathbb{R}$, ou les nombres complexes $\mathbb{C}$.

\subsection{Base d'un espace vectorielle}
\noindent
Soit un espace vectorielle $V$ et $\alpha_1, \alpha_2, \ldots, \alpha_n$ des scalaires.

\subsubsection{Combinaison linéaire}
\noindent
Soit $v_1, v_2, \ldots, v_n \in V$
\begin{definition}
      $\alpha_1 v_1 + \alpha_2 v_2 + \ldots + \alpha_n v_n$ est une combinaison linéaire de
      $v_1, v_2, \ldots, v_n$
\end{definition}

\subsubsection{Ensemble générateur}
\begin{definition}
      Un ensemble $S = \{ u_1, u_2, \ldots, u_n \} \subset V$, est un ensemble générateur si
      \[ \forall v \in V  \ \exists \alpha_1, \ldots, \alpha_n \text{ scalaire t.q. } v = \alpha_1 u_1 + \ldots + \alpha_n u_n\]
\end{definition}
\noindent
Note: Pour prouver qu'un ensemble est un ensemble générateur, il faut habituellement
prendre un élément général de l'espace vectorielle et exprimer cet élément général
comme une combinaison linéaire des vecteurs dans $S$.

\subsubsection{Indépendance linéaire}
\begin{definition}
      Un ensemble $S = \{ u_1, u_2, \ldots, u_n \} \subset V$, est un linéaire indépendant si
      \[
            \alpha_1 u_1 + \alpha_2 u_2 + \ldots + \alpha_n u_n = 0_v \implies \alpha_1 = \alpha_2 = \ldots = \alpha_n = 0
      \]
      sinon $S$ est linéairement dépendant, ou lié.
\end{definition}

\subsubsection{Déterminer l'indépendance linéaire de \texorpdfstring{$S \subset \R^n$}{S subset of Rn}}
\noindent
Soit $S = \{ u_1, u_2, \ldots, u_m \} \subset \R^n$ avec $u_j = \begin{pmatrix}
            u_{1j} & u_{2j} & \dots  & u_{nj} \end{pmatrix}^T, \ u_{kj} \in \R$ et $|S| = m$ \\
Considérons la matrice $M = \begin{pmatrix} u_1 & u_2 & \ldots & u_m \end{pmatrix}$. \\
Pour déterminer l'indépendance linéaire de $S$ à partir de $M$, il faut
échelonner la matrice $M$ ou la matrice $M^T$. En échelonnant $M$ ou $M^T$, on peut arriver 
à deux conclusions, soit \begin{enumerate}[1.]
      \item $\text{rg}(M^T) < m$ ou $\text{rg}(M) < m \implies S$ est lié.
      \item $\text{rg}(M^T) = m$ ou $\text{rg}(M^T) = m \implies S$ est linéairement indépendant.
\end{enumerate}
Note: Échelonner $M^T$ donne des résultats plus simple à interpréter puisque 
$0 \leq \text{rg}(M^T) \leq m$. Alors si $M^T$ échelonné contient une ligne nulle
on sait automatiquement que $S$ est lié, ce qui n'est pas le cas si $M$ échelonné
contient une ligne nulle.

\subsubsection{Base d'un espace vectorielle}
\begin{definition}
      Un ensemble $B = \{ u_1, u_2, \ldots, u_n \} \subset V$, est une base de $V$ si
      $B$ est un ensemble générateur de $V$ et $B$ est linéairement indépendant.
\end{definition}
\begin{definition}
      La dimension de $V$ est $\dim_F V = |B|$, ou $|B|$ est le nombre d'élément dans la base $B$
      de $V$ et $F$ est le corps de l'espace vectorielle.
\end{definition}
\begin{theorem}
      Le nombre de vecteur dans une base de $V$ ne dépend pas de la base choisi.
\end{theorem}
\begin{remark}
      Le théorème qui précède nous permet de définir la dimension de $V$ comme étant le
      nombre de vecteurs dans une base de $V$, puisque toutes les bases de $V$ contiennent
      le même nombre de vecteurs.
\end{remark}
\paragraph{Base canonique:}
Un espace vectorielle a souvent une base canonique, soit une base plus naturel à utiliser.
Par exemple, la base canonique de $\R^n$ est $S = \{ e_1, e_2, \ldots, e_n \}$.
On peut donc dire que $\dim_\R \R^n = n$

\subsubsection{Représentation d'un vecteur dans une base}
\noindent
Soit $V$ un espace vectorielle avec $\dim_F V = n$, et $B = \{ u_1, u_2, \ldots, u_n \}$ une base de $V$. \\
Cela veut dire que $ \forall \ v \in V  \ \exists \ \alpha_1, \ldots, \alpha_n \text{ scalaire t.q. } v = \alpha_1 u_1 + \ldots + \alpha_n u_n$
\begin{definition}
      Les scalaires $\alpha_1, \ldots, \alpha_n$ sont appellés les coordonnées de $v$ dans la base $B$.
\end{definition}
\begin{definition}
      La représentation de $v$ dans la base $B$ est noté $[v]_B$ et est exprimé
      \[
            [v]_B = \begin{pmatrix}
                  \alpha_1 \\
                  \alpha_2 \\
                  \vdots   \\
                  \alpha_n
            \end{pmatrix} \in F^n
      \]
\end{definition}

\subsubsection{Propriétés d'une base d'un espace vectorielle}
\begin{enumerate}[a)]
      \item Soit $v_1, v_2 \in V$, alors $v_1 = v_2 \iff [v_1]_B = [v_2]_B$
      \item Soit $S = \{ u_1, u_2, \ldots, u_n \} \subset V$ \\
            Si $S^\prime = \{ [u_1]_B, [u_2]_B, \ldots, [u_n]_B \}$ est linéairement indépendant alors $S$ l'est aussi
      \item La représentation dans une base est linaire, ce qui veut dire qu'elle satisfait ces deux propriétés:
            \begin{enumerate}[1.]
                  \item $[v_1 + v_2]_B = [v_1]_B + [v_2]_B$
                  \item $[\alpha v]_B = \alpha [v]_B$
            \end{enumerate}
\end{enumerate}