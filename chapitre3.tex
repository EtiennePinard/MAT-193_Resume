\unnumsec{Chapitre 3: Espace vectoriel de dimension finie}
\begin{definition}
      Un ensemble $V$ est un espace vectoriel sur un corps $F$
      si \begin{enumerate}[1)]
            \item $V$ est fermé sous l'addition c-à-d, $
                        \forall \ (v_1, v_2 \in V), \ v_1 + v_2 \in V
                  $
            \item $V$ est fermé sous la multication par un scalaire c-à-d, $
                        \forall \ (v \in V, \ \alpha \in F), \ \alpha v \in V
                  $
      \end{enumerate}
      Les éléments de $V$ sont appellés \guillemetleft \ vecteurs \guillemetright.
\end{definition}
\begin{remark}
      L'espace vectoriel le plus simple est $V = \{ 0_v \}$, ou $0_v$ est l'élément nulle.
\end{remark}
\begin{definition}
      Un scalaire $\alpha$ est un élément du corps $F$ associé à l'espace vectoriel.
\end{definition}
Dans notre cas, ce corps est soit les nombres réels $\mathbb{R}$, ou les nombres complexes $\mathbb{C}$.

\subsection{Base d'un espace vectoriel}
Soit un espace vectoriel $V$ et $\alpha_1, \alpha_2, \ldots, \alpha_n$ des scalaires.

\subsubsection{Combinaison linéaire}
Soit $v_1, v_2, \ldots, v_n \in V$
\begin{definition}
      $\alpha_1 v_1 + \alpha_2 v_2 + \ldots + \alpha_n v_n$ est une combinaison linéaire de
      $v_1, v_2, \ldots, v_n$
\end{definition}

\subsubsection{Ensemble générateur}
\begin{definition}
      Un ensemble $S = \{ u_1, u_2, \ldots, u_n \} \subset V$, est un ensemble générateur si
      \[ \forall v \in V  \ \exists \alpha_1, \ldots, \alpha_n \text{ scalaire t.q. } v = \alpha_1 u_1 + \ldots + \alpha_n u_n\]
\end{definition}
Note: Pour prouver qu'un ensemble est un ensemble générateur, il faut habituellement
prendre un élément général de l'espace vectoriel et exprimer cet élément général
comme une combinaison linéaire des vecteurs dans $S$.

\subsubsection{Indépendance linéaire}
\begin{definition}
      Un ensemble $S = \{ u_1, u_2, \ldots, u_n \} \subset V$, est un linéaire indépendant si
      \[
            \alpha_1 u_1 + \alpha_2 u_2 + \ldots + \alpha_n u_n = 0_v \implies \alpha_1 = \alpha_2 = \ldots = \alpha_n = 0
      \]
      sinon $S$ est linéairement dépendant, ou lié.
\end{definition}

\subsubsection{Déterminer l'indépendance linéaire de \texorpdfstring{$S \subset \R^n$}{S subset of Rn}}
Soit $S = \{ u_1, u_2, \ldots, u_m \} \subset \R^n$ avec $u_j = \begin{pmatrix}
            u_{1j} & u_{2j} & \dots  & u_{nj} \end{pmatrix}^T, \ u_{kj} \in \R$ et $|S| = m$ \\
Considérons la matrice $M = \begin{pmatrix} u_1 & u_2 & \ldots & u_m \end{pmatrix}$. \\
Pour déterminer l'indépendance linéaire de $S$ à partir de $M$, il faut
échelonner la matrice $M$ ou la matrice $M^T$. En échelonnant $M$ ou $M^T$, on peut arriver
à deux conclusions, soit \begin{enumerate}[1.]
      \item $\text{rg}(M^T) < m$ ou $\text{rg}(M) < m \implies S$ est lié.
      \item $\text{rg}(M^T) = m$ ou $\text{rg}(M^T) = m \implies S$ est linéairement indépendant.
\end{enumerate}
Note: Échelonner $M^T$ donne des résultats plus simple à interpréter puisque
$0 \leq \text{rg}(M^T) \leq m$. Alors si $M^T$ échelonné contient une ligne nulle
on sait automatiquement que $S$ est lié, ce qui n'est pas le cas si $M$ échelonné
contient une ligne nulle.

\subsubsection{Base d'un espace vectoriel}
\begin{definition}
      Un ensemble $B = \{ u_1, u_2, \ldots, u_n \} \subset V$, est une base de $V$ si
      $B$ est un ensemble générateur de $V$ et $B$ est linéairement indépendant.
\end{definition}
\begin{definition}
      La dimension de $V$ est $\dim_F V = |B|$, ou $|B|$ est le nombre d'élément dans la base $B$
      de $V$ et $F$ est le corps de l'espace vectoriel.
\end{definition}
\begin{theorem}
      Le nombre de vecteur dans une base de $V$ ne dépend pas de la base choisi.
\end{theorem}
\begin{remark}
      Le théorème qui précède nous permet de définir la dimension de $V$ comme étant le
      nombre de vecteurs dans une base de $V$, puisque toutes les bases de $V$ contiennent
      le même nombre de vecteurs.
\end{remark}
\paragraph{Base canonique:}
Un espace vectoriel a souvent une base canonique, soit une base plus naturel à utiliser.
Par exemple, la base canonique de $\R^n$ est $S = \{ e_1, e_2, \ldots, e_n \}$.
On peut donc dire que $\dim_\R \R^n = n$

\subsubsection{Représentation d'un vecteur dans une base}
Soit $V$ un espace vectoriel avec $\dim_F V = n$, et $B = \{ u_1, u_2, \ldots, u_n \}$ une base de $V$. \\
Cela veut dire que $ \forall \ v \in V  \ \exists \ \alpha_1, \ldots, \alpha_n \text{ scalaire t.q. } v = \alpha_1 u_1 + \ldots + \alpha_n u_n$
\begin{definition}
      Les scalaires $\alpha_1, \ldots, \alpha_n$ sont appellés les coordonnées de $v$ dans la base $B$.
\end{definition}
\begin{definition}
      La représentation de $v$ dans la base $B$ est noté $[v]_B$ et est exprimé
      \[
            [v]_B = \begin{pmatrix}
                  \alpha_1 \\
                  \alpha_2 \\
                  \vdots   \\
                  \alpha_n
            \end{pmatrix} \in F^n
      \]
\end{definition}

\subsubsection{Propriétés d'une base d'un espace vectoriel}
\begin{enumerate}[a)]
      \item Soit $v_1, v_2 \in V$, alors $v_1 = v_2 \iff [v_1]_B = [v_2]_B$
      \item La représentation dans une base est linaire, ce qui veut dire qu'elle satisfait ces deux propriétés:
            \begin{enumerate}[1.]
                  \item $[v_1 + v_2]_B = [v_1]_B + [v_2]_B$
                  \item $[\alpha v]_B = \alpha [v]_B$
            \end{enumerate}
      \item Soit $S = \{ u_1, u_2, \ldots, u_n \} \subset V$ \\
            Si $S^\prime = \{ [u_1]_B, [u_2]_B, \ldots, [u_n]_B \}$ est linéairement indépendant alors $S$ l'est aussi
\end{enumerate}

\subsection{Matrice de changement de base}

Soit $V$ un espace vectoriel avec $\dim_F V = n$ \\
Soit $B = \left\{ u_1, u_2, \dots, u_n \right\}$ et $B^\prime = \left\{ u_1^\prime, u_2^\prime, \dots, u_n^\prime \right\}$ deux bases de $V$ 
\begin{definition}
      La matrice de changement (ou de passage) de la base $B^\prime$ à $B$ est noté
      $P^{B^\prime}_B$ et est donné par $P^{B^\prime}_B = \begin{bmatrix}
           [ u_1^\prime]_B & [u_2^\prime]_B & \dots & [u_n^\prime]_B
      \end{bmatrix}$. Les bases sont liées par la matrice de passage par l'équation 
      \[P^{B^\prime}_B [v]_{B^\prime} = [v]_B\]
\end{definition}
Note: Il a plusieurs notation pour la matrice de passage. La notation utilisée
dans cette définition est équivalente à $\underset{B^\prime \to B}{P}$
\begin{lemma}
      La matrice de passage de $B$ à $B^\prime$ est l'inverse de la matrice de passage de 
      $B^\prime$ à $B$, soit \[P^B_{B^\prime} = \left(P^{B^\prime}_B\right)^{-1}\]
\end{lemma}
\begin{lemma}
      La matrice de passage est unique.
\end{lemma}

\subsection{L'indépendance linéaire des functions}
Soit $V$ l'espace vectoriel de toutes les fonctions $f: \R \to \R$ avec des scalaire réels \\
La dimensions de $V$ est infinie et il n'y a pas de bases dans $V$ \\
Soit ${v_1, v_2, \dots, v_n} \subset V$ et $S = \text{span}\{v_1, v_2, \dots, v_n\}$. \\
Pour déterminer l'indépendance linéaire de $S$, considérons $F(x) = \alpha_1 v_1 + \alpha_2 v_2 + \dots + \alpha_n v_n$ \\
Supposons que $S$ est linéairement indépendant, c-à-d que $F(x) = 0 \ \forall \ x \in \R$ \\
Il faut ensuite prendre $x_1, x_2, \dots, x_n$ valeurs et évaluer $F(x_1), F(x_2), \dots, F(x_n)$. \\
Le résultats va être $n$ équations de la forme $\alpha_1 v_1(x_j)+ \alpha_2 v_2(x_j)+ \dots + \alpha_n v_n(x_j) = 0$ \\
\underline{Important:} Pour que la méthode fonctionne, il faut que les $n$ équations soit uniques. 
En d'autres termes, si $F(a) = F(b)$ alors $a$ et $b$ ne peuvent pas les deux faire parties des $n$ valeurs choisis. \\
Maintenant, il faut résoudre le système homogène $\begin{cases}
      \alpha_1 v_1(x_1)+ \alpha_2 v_2(x_1)+ \dots + \alpha_n v_n(x_1) = 0 \\
      \alpha_1 v_1(x_2)+ \alpha_2 v_2(x_2)+ \dots + \alpha_n v_n(x_2) = 0 \\
      \vdots \\
      \alpha_1 v_1(x_n)+ \alpha_2 v_2(x_n)+ \dots + \alpha_n v_n(x_n) = 0 \\
\end{cases}$ \\
La résolution du système va aboutir à deux possibilitées: \begin{enumerate}
      \item Le système homogène à seulement la solution nulle $\implies S$ est linéairement indépendant
      \item Le système homogène à seulement une infinité de solution. Dans ce cas, on doit construire
      une nouvelle fonction $\hat{F}(x) = \beta_1 v_1 + \beta_2 v_2 + \dots + \beta_n v_n$ ou $\beta_1, \beta_2, \dots, \beta_n$
      est une solution non-nulle du système. Il faut alors démontrer que $\hat{F}(x) = 0 \ \forall \ x \in \R$. 
      Si c'est le cas, alors on peut conclure que $S$ est lié.
\end{enumerate}

\subsection{Espaces vectoriels munis d'un produit scalaire}

\subsubsection{Produit scalaire réel}
Soit $V$ un espace vectoriel réel
\begin{definition}
      \label{scpr_Real}
      Un produit scalaire dans $V$ est une application $\myfunc{\scpr{\cdot}{\cdot}}{V \times V}{\R}{u, v \in V}{\scpr{u}{v}}$, t.q.
      \begin{align*}
            &1. \quad \scpr{u}{v} = \scpr{v}{u}& &2. \quad \scpr{u}{v + \alpha w} = \scpr{u}{v} + \alpha \scpr{u}{w}& &3. \quad \scpr{u}{u} \geq 0 \ \text{et} \ \scpr{u}{u} = 0 \iff u = 0&
      \end{align*}
\end{definition}
\begin{remark}
      Le produit scalaire réel n'est pas unique. 
\end{remark}
\begin{definition}
      Une matrice $A \in M_{n \times n}(\R)$ est définie positif si $x^T A x > 0 \ \forall \ x \neq 0, \ x \in R^n$
\end{definition}
\begin{lemma}
      Soit $x,y \in \R^n$ et $A \in M_{n \times n}(\R)$ symétrique et définie positif, alors tout produit scalaire dans $\R^n$ 
      peut être écris comme tel, $\scpr{x}{y} = x^T A y$. On note ce produit scalaire modifié $\scpr{u}{v}_A$
\end{lemma}
\begin{definition}
      Le cas ou $A = I \implies \scpr{x}{y} = x^T I y = x^T y$ est appellé le produit scalaire 
      canonique (ou euclédien) dans $\R^n$
\end{definition}

\subsubsection{Produit scalaire complexe}
Soit $V$ un espace vectoriel complexe. Le produit scalaire complexe est défini de manière
similaire au produit scalaire réel [\ref{scpr_Real}] à l'exception d'une restriction de plus.
\begin{definition}
      Un produit scalaire dans $V$ est une application $\myfunc{\scpr{\cdot}{\cdot}}{V \times V}{\C}{u, v \in V}{\scpr{u}{v}}$, t.q.
      \begin{align*}
            &1. \quad \scpr{u}{v} = \scpr{v}{u}^*& &2. \quad \scpr{u}{v + \alpha w} = \scpr{u}{v} + \alpha \scpr{u}{w}& &3. \quad \scpr{u}{u} \geq 0 \ \text{et} \ \scpr{u}{u} = 0 \iff u = 0&
      \end{align*}
\end{definition}
\begin{remark}
      La définition implique que $\scpr{\alpha v + \beta u}{w} = \alpha^* \scpr{v}{w} + \beta^* \scpr{u}{w}$
\end{remark}
\begin{lemma}
      Soit $x,y \in \C^n$ et $A \in M_{n \times n}(\C)$ hermitienne et définie positif, alors tout produit scalaire dans $\C^n$ 
      peut être écris comme tel, $\scpr{x}{y}_A = x^\dagger A y$.
\end{lemma}
\begin{definition}
      Le produit scalaire canonique dans $\C^n$ est $\scpr{x}{y} = x^\dagger y$
\end{definition}

\subsubsection{Orthogonalité entre deux vecteurs}
Soit $V$ un espace vectoriel muni d'un produit scalaire
\begin{definition}
      $x, y \in V$ sont orthogonales $\iff \scpr{x}{y} = 0$. Alors on note $x \perp y$.
\end{definition}
\begin{remark}
      L'orthogonalité entre deux vecteurs est toujours par rapport au produit scalaire utilisé
\end{remark}

\subsubsection{Norme d'un vecteur}
Soit $V$ un espace vectoriel muni d'un produit scalaire
\begin{definition}
      La norme de $x \in V$ par rapport au produit scalaire dans $V$ est $\norm{x} = \sqrt{\scpr{x}{x}}$
\end{definition}
\begin{remark}
      La norme d'un vecteur change par rapport au produit scalaire utilisé
\end{remark}
\begin{definition}
      Un vecteur de norme 1 est dit un vecteur unitaire.
\end{definition}
\begin{lemma}
      Tout vecteur $x$ peut être transformé en vecteur unitaire avec la formule $\hat{x} = \frac{x}{\norm{x}}$
\end{lemma}

\subsubsection{Distance entre deux vecteurs}
Soit $V$ un espace vectoriel muni d'un produit scalaire
\begin{definition}
      La distance entre $x, y \in V$ est $\text{dist}(x, y) = \norm{x - y}$
\end{definition}
\begin{remark}
      Encore une fois, la distance entre deux vecteurs change par rapport au produit scalaire utilisé
\end{remark}

\subsubsection{Inégalité du produit scalaire}
Soit $V$ un espace vectoriel muni d'un produit scalaire, $u, u \in V$
\begin{theorem}
      L'inégalité de Cauchy-Schwarz est $\left|\scpr{u}{v}\right| \leq \norm{u}\norm{v}$
\end{theorem}
\begin{theorem}
      L'inégalité du triangle est $\norm{u + v} \leq \norm{u} + \norm{v}$
\end{theorem}

\subsection{Base orthonormales}
Soit $V$ un espace vectoriel muni d'un produit scalaire, $\dim V = n$ \\
Soit $S = \{u_1, u_2, \dots, u_m\} \subset V$
\begin{definition}
      $S$ orthonormale $\iff$
      $u_{k_1} \perp u_{k_2} \ \forall \ k_1 \neq k_2$ et $\norm{u_k} = 1 \ \forall \ k = 1, 2, \dots, m$ 
\end{definition}
\begin{definition}
      Le delta de Kronecker est $\delta_{kj} = \begin{cases}
            0 & \text{si} \ k \neq j \\
            1 & \text{si} \ k = j \\
      \end{cases}$
\end{definition}
\begin{remark}
      Le delta de Kronecker pour $i, j \in \{0, 1, \dots, n\}$ est la matrice identité $n \times n$.
      De plus, on peut utiliser le delta de Kronecker pour simplifier la définition d'un ensemble orthonormé.
\end{remark}
\begin{definition}
      $S$ orthonormale $\iff \scpr{u_k}{u_j} = \delta_{kj} \ \forall \ u_k, u_j \in S$ 
\end{definition}
\begin{definition}
      $B$ est une base orthonormale (ou orthonormé) dans $V$ si $B$ est une base dans $V$ et 
      $B$ est un ensemble orthonormale.
\end{definition}
Soit $B$ une base orthonormale de $V$. Notons le produit scalaire dans $V$ comme $\scpr{\cdot}{\cdot}_V$
\begin{theorem}
      Pour $x, y \in V$, on a $\scpr{x}{y}_V = [x]_B^\dagger [y]_B = \scpr{[x]_B}{[y]_B}_{\C^n}$
\end{theorem}
\begin{remark}
      Il est possible de traduire tout produit scalaire dans $V$ au produit scalaire canonique de
      $\C^n$ à l'aide d'une base orthonormale de $V$.
\end{remark}

\subsection{Projection orthogonal}
Soit $R = \{u_1, u_2, \dots, u_m\} \subset V$ un ensemble orthonormale avec $\dim V = n > m$ \\
Considérons le sous-espace $W = \text{span}\{R\} \subset V$ qui a $R$ comme base orthonormale
\begin{definition}
      Soit $v \in V$, alors $S_W(v) = \sum\limits_{k = 1}^{m}\scpr{u_k}{v}{u_k}$
\end{definition}
\begin{theorem}
      Le vecteur $S_W(v) \in W$ et $S_W(v) \perp w$ $\forall \ v \in V, \ w \in W$
\end{theorem}

\subsection{Orthonormalisation de Gram-Schmidt}
Soit $W = \{w_1, w_2, \dots, w_m\}$ un ensemble linéairement indépendant. L'orthonormalisation 
de Gram-Schmidt va produire un ensemble orthonormale $S = \{u_1, u_2, \dots, u_m\}$ à partir
des vecteurs de $W$. 
\begin{align*}
      &\text{Étape} \ 1. \quad \text{Posons} \ u_1 = \frac{w_1}{\norm{w_1}}& &\text{Alors} \ u_1 \ \text{est unitaire} \\
      &\text{Étape} \ 2. \quad \text{Posons} \ v_2 = w_2 - S_{<u_1>}(w_2)& &\text{Alors} \ v_2 \perp u_1 \\
      &\text{Étape} \ 3. \quad \text{Posons} \ u_2 = \frac{v_2}{\norm{v_2}}& &\text{Alors} \ u_2 \ \text{est unitaire et} \ u_2 \perp u_1 \\
      \vdots \\
      &\text{Étape} \ 2m - 2. \quad \text{Posons} \ v_{m} = w_{m} - S_{<u_1, \dots, u_{m - 1}>}(w_{m})& &\text{Alors} \ v_{m} \perp \{u_1, \dots, u_{m - 1}\} \\
      &\text{Étape} \ 2m - 1. \quad \text{Posons} \ u_{m} = \frac{v_{m}}{\norm{v_{m}}} & &\text{Alors} \ u_{m} \ \text{est unitaire et} \ u_{m} \perp \{u_1, \dots, u_{m - 1}\}
\end{align*}
Le procédé va se terminer quand toutes les $2m - 1$ étapes seront faites. \\
Note: La notation $<u_1, \dots, u_j>$ est une façons plus courte d'écrire $\text{span}\{u_1, \dots, u_j\}$