\section*{Chapitre 3: Espace vectorielle de dimension finie}

\subsection*{Définition}
\begin{itemize}
    \item[] Un ensemble $V$ est un espace vectorielle sur un corps ($\mathbb{R}$, $\mathbb{C}$)
          si \begin{enumerate}[1)]
              \item $V$ est fermé sous l'addition c-à-d, \begin{equation*}
                        v_1, v_2 \in V \implies v_1 + v_2 \in V
                    \end{equation*}
              \item $V$ est fermé sous la multication par un scalaire c-à-d, \begin{equation*}
                        v \in V, \ \alpha \text{ scalaire} \implies \alpha v \in V
                    \end{equation*}
          \end{enumerate}
    \item[] Un scalaire est un élément du corps de l'espace vectorielle. Dans notre cas, ce corps
          est soit les nombres réels $\mathbb{R}$, ou les nombres complexes $\mathbb{C}$.
    \item[] L'espace vectorielle le plus simple est $V = \{ 0_v \}$, ou $0_v$ est
          l'élément nulle.
    \item[] Pour vérifier si un ensemble est un espace vectorielle, il suffit
          de vérifier si l'ensemble respectent les deux propriétés de fermeture.
\end{itemize}

\subsection*{Combinaisons linéaire}
\begin{itemize}
    \item[] \textbf{Définition} \begin{itemize}
              \item[] Soit un espace vectorielle $V$ avec $v_1, v_2, \ldots, v_n \in V$ et $\alpha_1, \alpha_2, \ldots, \alpha_n$ des scalaires.
              \item[] Alors $\alpha_1 v_1 + \alpha_2 v_2 + \ldots + \alpha_n v_n$ est une combinaison linéaire de
                    $v_1, v_2, \ldots, v_n$
          \end{itemize}
    \item[] \textbf{Ensemble générateur} \begin{itemize}
              \item[] Un ensemble $S = \{ u_1, u_2, \ldots, u_n \} \subset V$, est un ensemble générateur
                    si tout les éléments de $V$ peuvent être exprimé comme une combinaison linéaire des vecteurs
                    dans $S$.
              \item[] Pour prouver qu'un ensemble est un ensemble générateur, il faut habituellement
                    prendre un élément général de l'espace vectorielle et exprimer cet élément général
                    comme une combinaison linéaire des vecteurs dans $S$.
          \end{itemize}
    \item[] \textbf{Indépendance linéaire} \begin{itemize}
              \item[] Un ensemble $S = \{ u_1, u_2, \ldots, u_n \} \subset V$, est un linéaire indépendant
                    si \begin{equation*}
                        \alpha_1 u_1 + \alpha_2 u_2 + \ldots + \alpha_n u_n = 0_v \implies \alpha_1 = \alpha_2 = \ldots = \alpha_n = 0
                    \end{equation*}
                    en d'autres mots, que cette équation a seulement la solution nulle (trivial). Si
                    ce n'est pas le cas, alors $S$ est linéairement dépendant, ou lié.
              \item[] \textbf{Déterminer l'indépendance linéaire d'un ensemble $S \subset \mathbb{R}^n$} \begin{enumerate}[1), itemsep = 0.5em]
                        \item[] Soit $S = \{ u_1, u_2, \ldots, u_m \} \subset \mathbb{R}^n$ avec $u_j = \begin{pmatrix}
                                      u_{1j} \\
                                      u_{2j} \\
                                      \dots  \\
                                      u_{nj}
                                  \end{pmatrix}, \ u_{kj} \in \mathbb{R}$
                        \item[] Il a $m$ vecteurs dans $S$ et chaque vecteur à $n$ élément.
                        \item[] De plus, considérons la matrice $M = \begin{pmatrix}
                                      u_{11} & u_{12}      & \dots  & u_{1m} \\
                                      u_{21} & u_{22}      & \dots  & u_{2m} \\
                                      \vdots & \phantom{u} & \ddots & \vdots \\
                                      u_{n1} & u_{n2}      & \dots  & u_{nm}
                                  \end{pmatrix} = \begin{pmatrix}
                                      u_1 & u_2 & \ldots & u_m
                                  \end{pmatrix}$
                        \item[] Remarquons que $M$ à les vecteurs de $S$ commes colonnes.
                        \item La première méthode consiste à échelonner la matrice $M^T = \begin{pmatrix}
                                      (u_1)^T \\
                                      (u_2)^T \\
                                      \ldots  \\
                                      (u_m)^T
                                  \end{pmatrix}$. Ici $M^T$ à les vecteurs de $S$ commes lignes. En échelonnant
                              $M$, on peut arriver à deux conclusions, soit \begin{enumerate}[1.]
                                  \item La forme échelonné de $M^T$ contient une ligne nulle,
                                        alors $S$ est lié.
                                  \item La forme échelonné de $M^T$ ne contient pas de ligne nulle,
                                        alors $S$ est linéairement indépendant.
                              \end{enumerate}
                        \item La deuxième méthode consiste à échelonner la matrice $M$. En échelonnant $M$, on peut arriver à deux conclusions, soit \begin{enumerate}[1.]
                                  \item La forme échelonné de $M$ contient plus ou égale de ligne nulle qu'il a de vecteur dans $S$,
                                        alors $S$ est lié.
                                  \item La forme échelonné de $M$ contient moins de ligne nulle qu'il a de vecteur dans $S$,
                                        alors $S$ est linéairement indépendant.
                              \end{enumerate}
                    \end{enumerate}
          \end{itemize}
\end{itemize}

\subsection*{Base d'un espace vectorielle} \begin{itemize}

    \item[] \textbf{Définition} \begin{itemize}[itemsep = 0.5em]
              \item[] Un ensemble $S = \{ u_1, u_2, \ldots, u_n \} \subset V$, est une base de $V$ si
                    $S$ est un ensemble générateur de $V$ et $S$ est linéairement indépendant.
              \item[] On peut aussi définir la dimension de $V$ comme $\dim_C V = |S|$,
                    ou $|S|$ est le nombre d'élément dans $S$ et $C$ est le corps que l'espace vectorielle
                    est défini sur. Dans notre cas, $C = \mathbb{R}$ ou $C = \mathbb{C}$
              \item[] De plus, un espace vectorielle a souvent une base canonique, soit une base
                    plus naturel à utiliser. Par exemple, canonique de $\mathbb{R}^n$ est $S = \{ e_1, e_2, \ldots, e_n \}$. 
                    On peut donc dire que $\dim_\mathbb{R} \mathbb{R}^n = n$
          \end{itemize}
    \item[] \textbf{Représentation d'un vecteur dans une base} \begin{itemize}[itemsep = 0.5em]
              \item[] Soit $V$ un espace vectorielle avec $\dim V = n$
              \item[] Soit $B = \{ u_1, u_2, \ldots, u_n \}$ une base de $V$
              \item[] Cela veut dire que $ \forall v \in V  \ \exists \alpha_1, \ldots, \alpha_n \text{ scalaire t.q. } v = \alpha_1 u_1 + \ldots + \alpha_n u_n$
              \item[] Les scalaires $\alpha_1, \ldots, \alpha_n$ sont appellés les coordonnées de $v$ dans la base $B$.
              \item[] On peut donc représenter $v$ dans la base $B$ comme \begin{equation*}
                [v]_B = \begin{pmatrix}
                    \alpha_1 \\
                    \alpha_2 \\
                    \vdots \\
                    \alpha_n
                \end{pmatrix}
              \end{equation*}
            \item[] Donc on peut représenter un élément de $V$ comme un élément dans $\mathbb{R}^n$ avec l'aide d'une base
          \end{itemize}
    \item[] \textbf{Propriétés d'une base d'un espace vectorielle} \begin{enumerate}[1)]
        \item $v_1 = v_2 \iff [v_1]_B = [v_2]_B$
        \item Soit $S = \{ u_1, u_2, \ldots, u_n \} \subset V$ \\
        Si $S^\prime = \{ [u_1]_B, [u_2]_B, \ldots, [u_n]_B \}$ est linéairement indépendant alors $S$ l'est aussi 
    \end{enumerate}
\end{itemize}