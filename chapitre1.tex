\section{Chapitre 1: Nombres Complexes}

\subsection{Définition des nombres complexes}
Rappel: les unités de $\R$ sont $1$ et $-1$ 
\begin{definition}
    $i$ est l'unité imaginaire telle que $i^2 = -1$ 
\end{definition}
Alors, il est possible de former un nouveau ensemble qui a a les unités $1, -1, i$ et $\scalaire{\alpha} \in \R$
\begin{definition}
    L'ensemble qui a a les unités $1, -1, i$ et $\scalaire{\alpha} \in \R$ est appellé nombre complexes et est noté $\C$
\end{definition}

\subsection{Représentation des nombres complexes}
Soit $x, y, r, \theta \in \R$ et $z \in \C$. Il a trois façons principales de représenter $z$
\begin{enumerate}
    \item \underline{Forme algégrique (cartésienne)}: $z = x + iy$
    \item \underline{Forme polaire}: $z = r(\cos(\theta)+i\sin(\theta))$
    \item \underline{Forme exponentielle}: $z = re^{i\theta}$
\end{enumerate}

\subsection{Changement de forme: Algèbrique à polaire/exponentielle}
Soit $z = x + iy = r(\cos(\theta)+i\sin(\theta)) = re^{i\theta}$ 
\begin{definition}
    On appelle $r$ le module de $z$ noté $|z|$ avec $r = |z| = \sqrt{x^2 + y^2}$
\end{definition}
\begin{definition}
On appelle $\theta$ l'argument de $z$ noté $\text{Arg}(z)$ avec \[\theta = Arg(z) =
    \begin{cases}
        \arctan(\frac{y}{x})       & x > 0                            \\
        \arctan(\frac{y}{x}) - \pi & x < 0                            \\
        \text{sign}(y)\frac{\pi}{2}           & x = 0, \ \text{sign}(y) \ \text{est la signe de y}
    \end{cases}\]
\end{definition}
\begin{remark}
    L'argument non unique de $z$ est $\arg(z) = \text{Arg}(z) + 2\pi k, \ k \in \Z$
\end{remark}

\subsection{Opérations propres aux nombres complexes}
Soit $z \in \C, \ x, y \in \R, \ \text{t.q.} \ z = x + iy$ 
\begin{definition}
    Les parties réel et imaginaire de $z$ sont $\Real{z} = x \in \R$ et $\Ima{z} = y \in \R$
\end{definition}
\begin{definition}
    Le conjugée de $z$ est $z^* = \bar{z} = x - iy = \Real{z} - i\Ima{z}$
\end{definition}

\subsubsection{Propriétés des opérations}
Le conjugée et le module sont distributif sur l'addition, la multiplication et la division
\[
    \begin{matrix}
        \left(z_1 + z_2\right)^* = z_1^* + z_2^* & \left(z_1z_2\right)^* = z_1^{*} z_2^* & \left( \frac{z_1}{z_2}\right)^{*} = \frac{z_1^*}{z_2^*} \\[0.5em]
        \left|z_1 + z_2\right| = |z_1| + |z_2|   & \left|z_1z_2\right| = |z_1||z_2|      & \left| \frac{z_1}{z_2}\right| = \frac{|z_1|}{|z_2|}
    \end{matrix}
\]
Le conjugée de $z^*$ est $z$, soit $\left(z^*\right)^* = z$ \\
Le conjugée $z$ et le module de $z$ est relié par $zz^* = |z|^2$ \\
L'argument de $z$ à des propriétés semblabes aux propriétés logarithmiques
\[
    \begin{matrix}
        \arg(z_1z_2) = \arg(z_1) + \arg(z_2) & \arg\left( \frac{z_1}{z_2}\right) = \arg(z_1) - \arg(z_2)
    \end{matrix}
\]

\subsection{Racines entières}
Soit $z = re^{i\theta},\ n \in \N$.
\[
    \sqrt[n]{z} = \sqrt[n]{r} \left( e^{i \left( \frac{\theta + 2\pi k}{n} \right) } \right), \ k = 0, \dots, n - 1
\]
On peut aussi calculer les racines de $z$ de manières récursives. \\
Si $w$ est une racine de $z$, alors on a
\[
    w_k = \begin{cases}
        w_{k - 1}e^{ \frac{2\pi i}{n} }                    & k > 0 \\[0.5em]
        \sqrt[n]{r} \left( e^{i \frac{\theta}{n} } \right) & k = 0
    \end{cases}
\]
\begin{remark}
    Deux racines entières consécutives sont séparées par un angle de $\frac{2\pi}{n}$
\end{remark}

\subsection{Exponentielle  et logarithme}
Soit $z \in \C, \ x, y \in \R, \ \text{t.q.} \ z = x + iy$ \\
L'exponentielle de $z$ est
\[
    e^z = e^{x + iy} = e^x e^{iy} = e^x (cos(y) + isin(y))
\]
Le logarithme de $z$ est
\begin{align*}
    \ln(z) = \log(z) & = \ln\left(|z|e^{i(\theta + 2\pi k)}\right)      \\
                     & = \ln|z| + i(\theta + 2\pi k), \ k \in \Z        \\
                     & = \ln|z| + i(\text{Arg}(z) + 2\pi k), \ k \in \Z \\
                     & = \ln|z| + i\arg(z)                              \\
\end{align*}

\subsection{Puissance Complexe}
Soit $z \in \C$ et $w \in \C, \ x, y \in \R \ \text{t.q.} \ w = x + iy$ 
\[
    z^w = \left( e^{\ln(z)}\right)^{w} = e^{\ln(z)w}
    = e^{\left(\ln|z| + i\arg(z)\right)\left(x + iy\right)}
    = |z|^{x} e^{-y\arg(z)} e^{ i \left(x\arg(z) + y\ln|z|\right) }
\]
Si $y \neq 0$ alors $z^w$ prend une infinité de valeurs distinctes, puisque $e^{-y\arg(z)} = e^{-y\text{Arg}(z) + 2\pi yk}, \; k \in \Z$ \\
Si $y = 0 \implies w \in \R$ alors le nombre de valeurs de $z^w$ dépend dans quelle ensemble $x$ est. \\
Si $x \in \Q$, alors $x = \frac{m}{n}, \ m \in \Z, \ n \in \N \implies \exists \ k \in \Z \ \text{t.q.} \ kx \in \Z \implies z^w$ a $n$ valeurs distinctes \\
Si $x \in \R \backslash \Q \implies \not \exists \ k \in \Z \ \text{t.q.} \ kx \in \Z \implies z^w$ a une infinité de valeurs distinctes \\
Dans ce dernier cas, si $z, w \in \R$ alors, par convention, on prend $k = 0$ pour que $z^w \in \R$.

\subsection{Théorème fondamental de l'algèbre}
\label{theorem_fondamental_algebra}
Soit $p(x) = a_n x^n + a_{n-1} x^{n-1} + \dots + a_1 x + a_0, \; a_j \in \C, \; a_n \neq 0, \; n \in N$ \\
Alors on peut écrire $p(x)$ en terme de ses racines $\{x_1, x_2, \dots ,x_m\}, \; x_j \in \C $, soit
\[ p(x) = a_n(x - x_1)^{k_1} (x - x_2)^{k_2} \dots (x - x_m)^{k_m} \]
où $k_j$ est la multiplicité de la racine $x_j$ avec $k_1 + k_2 + \dots + k_m = n$ \\
Alors $p(x)$ a exactement $n$ racines complexes en comptant les multiplicités.