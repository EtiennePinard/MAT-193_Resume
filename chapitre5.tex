\unnumsec{Chapitre 5: Espace dual d'un espace vectoriel}

\subsection{Fonctionnelles linéaire}
Soit $V$ un espace vectoriel avec les scalaires $F$
\begin{definition}
    Une forme, ou une fonctionnelle linéaire est une application
    $f\colon V \to F$ qui, pour tout $u, v \in V$, $\scalaire{\alpha}$,
    satisfait les deux propriétés suivantes
    \begin{align*}
        1. \quad f(u + v) = f(u) + f(v)& &2. \quad f(\alpha v) = \alpha f(v)
    \end{align*}
\end{definition}
\begin{remark}
    La fonctionnelle linéaire la plus simple est $\myfunc{f}{V}{F}{v}{0}$
\end{remark}
\begin{definition}
    Soit $f_1, f_2$ deux fonctionnelles linéaires. \\
    Définissons leur somme $f_1 + f_2$ comme tel:
    $(f_1 + f_2)(v) = f_1(v) + f_2(v) \ \forall \ v \in V $ \\
    Définissons la multiplication par un scalaire $\alpha f_1$ comme tel: 
    $(\alpha f_1)(v) = \alpha f_1(v) \ \forall \ v \in V$
\end{definition}
\begin{lemma}
    L'ensembles des fonctionnelles linéaires sur $V$ est lui-même un espace vectoriel
\end{lemma}
\begin{definition}
    L'espace des fonctionnelles sur $V$ est appelé l'espace dual de $V$ et est noté $V^*$
\end{definition}
\begin{theorem}
    Soit $B = \{b_1, \dots, b_n\}$ une base dans $V$ avec $[v]_B = \begin{pmatrix}
        v_1 & \dots & v_n \end{pmatrix}^T \ v \in V$. 
    Alors la base de $V^*$ est $B^* = \{\epsilon_1, \dots, \epsilon_n\}$ avec 
    $\myfunc{\epsilon_j}{V}{F}{v}{v_j}$, ou $v_j$ est le jème coordonné de $v$ dans le base $B$.
\end{theorem}
\begin{remark}
    La définition de $\epsilon$ implique que $\epsilon_j(b_k) = \delta_{jk} \ \forall \ b_k \in B$
\end{remark}
\begin{corollary}
    $\dim V = n \implies \dim V^* = n$
\end{corollary}
\begin{corollary}
    $\left(\C^n\right)^* = \C^n$ et $\left(\R^n\right)^* = \R^n$ 
\end{corollary}

\subsection{Espace dual dans l'espace vectoriel muni d'un produit scalaire}
Soit $V$ un espace vectoriel. Pour $x \in V$ posons $\hat{x} = f_x$ ou $\myfunc{f_x}{V}{F}{v}{\scpr{x}{v}}$
\begin{definition}
    On appelle $x = f_x \in V^*$ le vecteur dual de $x$
\end{definition}
\begin{theorem}
    $\forall \ f \in V^* \ \exists! \ x \in V$ t.q. $f = f_x$
\end{theorem}
\subsubsection{Comment trouver le vecteur dual de \texorpdfstring{$f \in V^*$}{f in V star}}
Soit $B = \{b_1, \dots, b_n\}$ une base \underline{orthonormale par rapport à $\scpr{\cdot}{\cdot}_V$} dans $V$ et $f \in V^*$ \\
Posons $f(b_j) = a_j \in \C^n$, alors la représentation du vecteur dual $x$ à $f$ dans la base $B$ est donné par 
\[
[x]_B = \begin{pmatrix} (a_1)^* & \dots & (a_n)^* \end{pmatrix}^T 
\]
Prendre note que dans le cas ou le produit scalaire
dans $V$ n'est pas le produit scalaire canonique, il est souvent plus simple d'utiliser une 
méthode plus directe que de trouver une base orthonormale par rapport au produit scalaire dans $V$.
\begin{definition}
    Soit $B = \{u_1, \dots, u_n\}$ une base dans $V$, alors 
    une base $B^* = \{\epsilon_1, \dots, \epsilon_n\}$ dans $V^*$ est dite base dual à 
    $B \iff \epsilon_j(u_k) = \delta_{jk}$ 
\end{definition}
\subsection{Notation de Dirac}
Soit $V$ un espace vectoriel muni d'un produit scalaire
\begin{definition}
    Dans la notation de Dirac, un élément $v \in V$ est noté $\ket{v} \in V$ et $\ket{\cdot}$ est appelé un ket. 
    Le vecteur dual de $\ket{v}$ est noté $\bra{v}$ et $\bra{\cdot}$ est appelé un bra. L'évaluation de 
    du vecteur dual $f_v = \ket{v}$ avec un vecteur $x \in V$ est noté $f_v(x) = f_v(\ket{x}) = \braket{v}{x}$
\end{definition}